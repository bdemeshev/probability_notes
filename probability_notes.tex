\documentclass{article}
\usepackage{sidenotes}
% \usepackage{libertine}

\usepackage{csquotes}

% TODO: рассказать про функции производящие множества прямо при введении множества исходов



\usepackage{hyperref}
\hypersetup{
    colorlinks=true,
    linkcolor=blue,
    filecolor=magenta,      
    urlcolor=cyan,
    pdftitle={Overleaf Example},
    pdfpagemode=FullScreen,
    }

\usepackage{tikzducks}

\usepackage{tikz} % картинки в tikz
\usepackage{microtype} % свешивание пунктуации

\usepackage{array} % для столбцов фиксированной ширины

\usepackage{indentfirst} % отступ в первом параграфе

\usepackage{sectsty} % для центрирования названий частей
\allsectionsfont{\centering}

\usepackage{amsmath, amsfonts, amssymb, amsthm} % куча стандартных математических плюшек


\usepackage{comment}

\usepackage[left=1.5cm, right=6cm, marginparwidth=4.5cm, bottom=2cm]{geometry}

%\usepackage[top=2cm, left=1.2cm, right=1.2cm, bottom=2cm]{geometry} % размер текста на странице

\usepackage{lastpage} % чтобы узнать номер последней страницы

\usepackage{enumitem} % дополнительные плюшки для списков
%  например \begin{enumerate}[resume] позволяет продолжить нумерацию в новом списке
\usepackage{caption}

\usepackage{url} % to use \url{link to web}

%\usepackage{fancyhdr} % весёлые колонтитулы
%\pagestyle{fancy}
%\lhead{}
%\chead{}
%\rhead{Домашние задания для самураев}
%\lfoot{}
%\cfoot{}
%\rfoot{}

%\renewcommand{\headrulewidth}{0.4pt}
%\renewcommand{\footrulewidth}{0.4pt}

\usepackage{tcolorbox} % рамочки!

\usepackage{todonotes} % для вставки в документ заметок о том, что осталось сделать
% \todo{Здесь надо коэффициенты исправить}
% \missingfigure{Здесь будет Последний день Помпеи}
% \listoftodos - печатает все поставленные \todo'шки


% более красивые таблицы
\usepackage{booktabs}
% заповеди из докупентации:
% 1. Не используйте вертикальные линни
% 2. Не используйте двойные линии
% 3. Единицы измерения - в шапку таблицы
% 4. Не сокращайте .1 вместо 0.1
% 5. Повторяющееся значение повторяйте, а не говорите "то же"


\setcounter{MaxMatrixCols}{20}
% by crazy default pmatrix supports only 10 cols :)


\usepackage{fontspec}
\usepackage{libertine}
\usepackage{polyglossia}

\setmainlanguage{russian}
\setotherlanguages{english}

% download "Linux Libertine" fonts:
% http://www.linuxlibertine.org/index.php?id=91&L=1
% \setmainfont{Linux Libertine O} % or Helvetica, Arial, Cambria
% why do we need \newfontfamily:
% http://tex.stackexchange.com/questions/91507/
% \newfontfamily{\cyrillicfonttt}{Linux Libertine O}

\AddEnumerateCounter{\asbuk}{\russian@alph}{щ} % для списков с русскими буквами
\setlist[enumerate, 2]{label=\asbuk*),ref=\asbuk*}

\graphicspath{{./figures}}

%% эконометрические сокращения
\DeclareMathOperator{\IMR}{IMR} % Inverse Mill's ratio
\DeclareMathOperator{\Cov}{\mathbb{C}ov}
\DeclareMathOperator{\Corr}{\mathbb{C}orr}
\DeclareMathOperator{\sCov}{sCov}
\DeclareMathOperator{\sCorr}{sCorr}
\DeclareMathOperator{\sVar}{sVar}
\DeclareMathOperator{\Var}{\mathbb{V}ar}
\DeclareMathOperator{\hVar}{\widehat{\Var}}
\DeclareMathOperator{\hCov}{\widehat{\Cov}}
\DeclareMathOperator{\hCorr}{\widehat{\Corr}}
\DeclareMathOperator{\col}{col}
\DeclareMathOperator{\se}{se}
\DeclareMathOperator{\row}{row}
\DeclareMathOperator{\grad}{grad}
\DeclareMathOperator{\BestLin}{BestLin}

\let\P\relax
\DeclareMathOperator{\P}{\mathbb{P}}

\DeclareMathOperator{\E}{\mathbb{E}}
\DeclareMathOperator{\trace}{trace}
\DeclareMathOperator{\rank}{rank}
\DeclareMathOperator{\tr}{\trace}
\DeclareMathOperator{\rk}{\rank}
\DeclareMathOperator{\card}{card}
\DeclareMathOperator{\Span}{Span}
\DeclareMathOperator{\mgf}{mgf}

\DeclareMathOperator{\Convex}{Convex}
\DeclareMathOperator{\plim}{plim}


\usepackage{mathtools}
\DeclarePairedDelimiter{\norm}{\lVert}{\rVert}
\DeclarePairedDelimiter{\abs}{\lvert}{\rvert}
\DeclarePairedDelimiter{\scalp}{\langle}{\rangle}
\DeclarePairedDelimiter{\ceil}{\lceil}{\rceil}



\newcommand{\cN}{\mathcal{N}}
\newcommand{\cF}{\mathcal{F}}

\newcommand{\RR}{\mathbb{R}}
\newcommand{\NN}{\mathbb{N}}

\newcommand{\dBern}{\mathrm{Bern}}
\newcommand{\dPois}{\mathrm{Pois}}
\newcommand{\dBin}{\mathrm{Bin}}
\newcommand{\dMult}{\mathrm{Mult}}
\newcommand{\dGeom}{\mathrm{Geom}}
\newcommand{\dNHGeom}{\mathrm{NHGeom}}
\newcommand{\dHGeom}{\mathrm{HGeom}}
\newcommand{\dDUnif}{\mathrm{DUnif}}
\newcommand{\dFS}{\mathrm{FS}}
\newcommand{\dNBin}{\mathrm{NBin}}

\newcommand{\dTri}{\mathrm{Triangle}}
\newcommand{\dUnif}{\mathrm{Unif}}
\newcommand{\dCauchy}{\mathrm{Cauchy}}
\newcommand{\dN}{\mathcal{N}}
\newcommand{\dLN}{\mathcal{LN}}
\newcommand{\dExpo}{\mathrm{Expo}}
\newcommand{\dBeta}{\mathrm{Beta}}
\newcommand{\dGamma}{\mathrm{Gamma}}
\newcommand{\dWei}{\mathrm{Wei}}
\newcommand{\dLogistic}{\mathrm{Logistic}}
\newcommand{\dRayleigh}{\mathrm{Rayleigh}}
\newcommand{\dPareto}{\mathrm{Pareto}}




\newcommand{\dist}{\text{dist}}




\usepackage{thmtools} % тонкая настройка окружения теорем

\declaretheorem[
style=definition, 
title=Теорема, 
numberwithin=section, 
]{theorem}


\declaretheorem[
style=definition, 
title=Метатеорема, 
numberwithin=section, 
]{metatheorem}


\declaretheorem[
style=definition, 
title=Определение, 
sibling=theorem, 
]{definition}



\title{Caesar Quick start}
\author{Andy Thomas}
% \publisher{Bielefeld University}

\begin{document}
% no page numbering in front matter
%\frontmatter
% generate the title page
% \maketitlepage
% show the table of contents
\tableofcontents
% start to number the pages
%\mainmatter
% The first chapter with annotation and citations
\section{Quick start}
%


При определении функции плотности рассказывать единицы измерения. 
Длина пойманного удава, случайная величина измеряется в метрах, $dx$ измеряется в метрах. 
Вероятность — безразмерная, значит плотность должна быть в метрах$^{-1}$.

Бета распределение логично вводить после равномерного на отрезке. 

А второй раз — как отношение гамма-распределений. 

И бета и гамма удобнее сначала ввести в качестве частного случая с помощью техники о-малых. 


Метод Ван-Гога! Как правильно отрезать уши на графе?

Задача про камень-ножницы-бумагу. Задача про число шестёрок при парных бросках кубика.
Что будет раньше?

От аксиом Гершеля-Максвелла легко выйти на нормальность линейной комбинации нормальных!
Берём и дополняем нужную нам линейнейную комбинацию до ортонормированного базиса! Всё!


Давать два определения ковариации! И дисперсии!

\url{https://math.stackexchange.com/questions/2423658/covariance-of-increasing-functions-of-random-variables}

Одно — с парами величин и делением на два!


Гибридная функция плотности (название?) для пары дискретная-непрерывная. Тут можно много задач создать!
\url{https://math.stackexchange.com/questions/887128/find-the-distribution-when-parameter-is-random}
\url{https://math.stackexchange.com/questions/1622003/joint-cdfs-of-both-continuous-and-discrete-random-variables}



Мартингал в дискретном времени лучше определять через $\E(M_{t+1} \mid \cF_t)$, а равенство для $\E(M_{t+h} \mid \cF_t)$ лучше доказывать как свойство. 
Слабым студентам проще понять, что достаточно проверить мартингальное свойство только для шага в единицу времени. 




\begin{definition}[Случайная величина]\label{def:random-variable}\index{случайная величина}
\textit{Случайная величина} — это функция из множества исходов эксперимента $\Omega$ в множество действительных чисел $\RR$.
\end{definition}

По европейской традиции случайные величины обычно обозначают с помощью заглавных букв,
в российской традиции обычно используют греческие буквы.

Если случайная величина моделирует время наступления какого-то события, 
то с помощью особого значения $+\infty$ описывают сценарий, в котором событие не наступило. 
В этом случае множеством значений случайной величины может быть $\{1, 2, 3, \dots\} \cup \{\infty\}$.
    

\begin{theorem}[Критерий независимости]\label{thm:rv-independency}\index{критерей независимости случайных величин}
Случайные величины $X$ и $Y$ независимы, если и только если события $A = \{X \leq x\}$ и $B = \{Y \leq y\}$
независимы для любых действительных чисел $x$ и $y$.
\end{theorem}

\begin{theorem}[Критерий независимости]\label{thm:rv-independency-expectation}\index{критерей независимости случайных величин}
Случайные величины $X$ и $Y$ независимы, если и только если для любых\sidenote{Для которых левая и правая части формулы определены.} действительных функций $f$ и $g$
\[
\E(f(X)\cdot g(Y)) = \E(f(X)) \cdot \E(g(Y)).
\]
\end{theorem}
    
В частности, из критерия следует, что для независимых случайных величин $X$ и $Y$ верно равенство $\E(XY) = \E(X)\E(Y)$.




\begin{theorem}[Критерий независимости дискретных величин]\label{thm:rv-independency-discrete}\index{критерей независимости случайных величин}
Дискретные случайные величины $X$ и $Y$ независимы, если и только если события $A = \{X = x\}$ и $B = \{Y = y\}$
независимы для любых действительных чисел $x$ и $y$.
\end{theorem}


\begin{metatheorem}[О доказательствах про дискретные величины.]\label{mthm:discrete-proofs}\index{метатеорема о доказательствах про дискретные величины}
Все доказательства утверждений о дискретных случайных величинах доказываются перегруппировкой слагаемых. 
\end{metatheorem}


\begin{definition}\label{def:binomial-distribution}\index{биномиальное распределение}
Монета\marginnote{Биномиальное распределение, $X \sim \dBin(n, p)$} выпадает решкой с вероятностью $p$ при каждом броске независимо от других бросков. 
Мы подбрасываем монету $n$ раз. 

Случайная количество выпавших решек $X$ имеет \emph{биномиальное распределение} с параметрами $n$ и $p$.
\end{definition}
    

\begin{definition}\label{def:negbinomial-distribution}\index{биномиальное распределение}
У\marginnote{Отрицательное биномиальное распределение, $X \sim \dNBin(r, p)$} нас есть монета, выпадающая решкой с вероятностью $p$ при каждом броске независимо от других. 
Мы подбрасываем монету до выпадения $r$ орлов.   
    
Случайная количество выпавших решек $X$ имеет \emph{отрицательное биномиальное распределение} с параметрами $r$ и $p$.
\end{definition}
    


\section*{Сходимости случайных величин}


\begin{definition}\label{def:convergence-in-distribution}\index{сходимость по распределению}
Последовательность случайных величин $(R_n)$ сходится\marginnote{$R_n \overset{\dist}{\to} R$} к случайной величине $R$,
если 
\[
\lim \P(R_n \in A) = \P(R \in A)
\]
для любого подмножества $A \subset \RR$ с нулевой вероятностью попадания $R$ в его край, $\P(R \in dA) = 0$. 
\end{definition}
    
Классическое определение из типичного учебника немного отличается по формулировке:
\begin{definition}\label{def:convergence-in-distribution2}\index{сходимость по распределению}
Рассмотрим последовательность случайных величин $R_n$ с функциями распределения $F_n(x)$ и случайную величину $R$ с функцией распределения $F(x)$.
Последовательность случайных величин $(R_n)$ сходится\marginnote{$R_n \overset{\dist}{\to} R$} к случайной величине $R$,
если 
\[
\lim F_n(x) = F(x)
\]
во всех точках непрерывности функции $F()$.
\end{definition}

Определения полностью эквивалентны.
\begin{proof}
Заметим, что классическое определение получается из первого в виде частного случая. 
Если рассмотреть множество $A$ вида $A = (-\infty, x]$, то граница $A$ — это просто одна точка, $dA = \{x\}$.
В этом случае условие $\P(R \in dA) = 0$ превращается в непрерывность функции $F$ в точке $x$.

В обратную сторону доказательство более техническое и следует из того, что любое\marginnote{для которого вероятность $\P(R \in A)$ определена} множество $A$ можно сконструировать 
из множеств вида $(-\infty, x]$.
При конструировании можно не обращать внимания на границу, в силу требования $\P(R \in dA) = 0$. 
Например, 
\[
[5; 6] = (-\infty, 6] \ (-\infty, 5] \cup \{5\}. 
\]
\end{proof}


\section{Экспоненциальное распределение}

Представим себе Нестареющего Кащея. 
Он смертен, но исключительно из-за внешних причин, не связанных со своим здоровьем. 
Скажем, какой-нибудь Иван-Дурак может случайно сломать иглу, спрятанную в яйце. 
Продолжительность жизни Нестареющего Кащея обозначим величиной $X$.
Если Нестареющий Кащей проживёт $s$ лет, то это никак не изменит его шансы прожить ещё дополнительно $t$ лет.
Другими словами, условная вероятность $\P(X > t + s \mid X > s)$ зависит только от $t$ и не зависит от $s$. 
Можно записать это свойство нестарения Кащея в виде уравнения: 
\[
\P(X > t + s \mid X > s) = \P(X > t) \text{ для любых } s, t > 0.
\]
В реальности это свойство бывает выполнено примерно. 
Если лампочка проработает ещё сутки, то её шансы перегореть за следующий час практически не изменятся. 
Свойство нестарения или медленного старения встречается довольно часто, и для него существует специальный термин. 

\begin{definition}[Экспоненциальное распределение, отсутствие памяти]\label{def:exponential-distribution}\index{экспоненциальное распределение}
Случайная величина $X$ имеет \emph{экспоненциальное распределение}, если $X$ удовлетворяет \emph{условию отсутствия памяти}
\[
 \P(X > t + s \mid X > s) = \P(X > t) \text{ для любых } s, t > 0.
\]    
\end{definition}


\section{Пуассовский поток}



\section{CLT}

Хорошее изложение характеристических функций

\url{https://web.ma.utexas.edu/users/gordanz/notes/characteristic.pdf}

clt

\url{https://sas.uwaterloo.ca/~dlmcleis/s901/chapt6.pdf}



\section{Толерантные интервалы?}

whuber general comments

\url{https://stats.stackexchange.com/questions/26702/prediction-and-tolerance-intervals}

good text
\url{https://www.math.kth.se/matstat/gru/sf2955/tolerans.pdf}

Howard Gitlow, 
Intro Stats Students Need Both Confidence and Tolerance (Intervals)






\end{document}
We compiled a minimal example file to show the basic use of the caesar class, which allows the typesetting of (science) textbooks or theses. The class itself is a reference implementation of the \emph{sidenotes} package. The package provides the additional functionality\sidenote{namely the use of marginal material such as this note or even figures or tables.} and the class gives sensible default values for page margins, chapter formatting and such. The caesar class is derived from the standard \LaTeX-book class and a little bit of experience with the standard class might be very helpful. In this example, biblatex is used for the references.

The first pages of the book (the frontmatter) are not numbered, the numbering starts after the \texttt{mainmatter} macro, which is called after the generation of the title page. The layout has ample margins to allow annotations. A main feature and the package is the sidenote, which is a footnote in the margin and can be placed with the \texttt{sidenote} macro.\sidenote{All information is on the same page, no turning of pages is necessary.} It is very similar to \texttt{footnote} and tries to emulate its behavior. The sidenote moves up or down (floats) to not overlap with other floats in the margin and all the sidenotes are subsequently numbered. 

References can be put in the margin as well.
%\sidecite[For the ideas behind all this, please see:][ and more work by Tufte.]{Tufte1990,Tufte2006} 
The macro was named \texttt{sidecite} and is defined with two optional parameters (prefix and postfix) similar to \texttt{cite} taken from the biblatex package. 
The next two sections describe the different options for the use of figures and tables in a document. We start with a couple of figures.
% A section with a couple of figures
\section{Figures}
%
\begin{marginfigure}%
    \includegraphics[width=\marginparwidth]{example-image-a}%
    \caption{A small rectangle put in the margin.\label{rectangle}}%
\end{marginfigure}%
%
There are three basic options to include figures in a document. The first option is a small figure and its caption in the margin. Figure \ref{rectangle} shows that with a gray rectangle framing the letter \emph{A}. We simply use the \texttt{marginfigure} environment instead of the \texttt{figure} one. 

The next alternative is a figure in the text frame. The figure is placed using the regular \LaTeX-figure environment and its  caption, which is displayed in figure~\ref{rectangle2}. 
%
\begin{figure}[htbp]%
	\includegraphics[height=180pt,width=\textwidth]{example-image-b}%
	\caption{A larger rectangle in the main area of the text, i.e.\ it does not span into the margin.}%
	\label{rectangle2}%
\end{figure}%
%

In case that a wider figure is needed, the third option spans over the text as well as the margin area. Here, the common \texttt{figure*} environment can be used. The figure options make it easy to choose the appropriate size for a given input file. 
%
\begin{figure*}[htbp]
    \includegraphics[height=180pt,width=400pt]{example-image-c}%
    \caption{An even larger rectangle. This is the widest figure option. Both, the text as well as the margin width are used for the diagram.}
    \label{rectangle3}
\end{figure*}
%

% Next section with a variety of tables
\section{Tables}
The same set of options (small, normal and wide) are also available for tables. The first option is a small table in the margin, this \texttt{margintable} is shown in table \ref{table1}.
%
\begin{margintable}%
	\begin{tabular}{lll}%
     A&B&C\\%
     0.50&0.47&0.48\\%
    \end{tabular}%
	\vspace{2pt}
	\caption{A couple of numbers in a table in the margin.\label{table1}}%
\end{margintable}%

Table \ref{table2} displays a larger table with more numbers. This is done using regular \LaTeX-macros for placing the table along with its caption. 
%
\begin{table}[htbp]%
	 \begin{tabular}{lllllllll}%
     A&B&C&D&E&F&G&H&I\\%
    0.50&0.47&0.48&0.50&0.47&0.48&0.60&0.39&1.00\\%
    \end{tabular}%
	\vspace{2pt}%
	\captionsetup{width=\textwidth, justification=justified}%
	\caption{A couple of numbers in a larger table. This table spans the usual text width.\label{table2}}%
\end{table}%

The \texttt{table*} environment is also defined in analogy to \texttt{figure*} and is demonstrated in table \ref{table3}.
%
\begin{table*}[h!]
 \begin{tabular}{lllllllllllll}%
     A&B&C&D&E&F&G&H&I&J&K&L&M\\%
    0.50&0.47&0.48&0.50&0.47&0.48&0.60&0.39&1.00&0.50&0.47&0.48&0.60\\%
  \end{tabular}%
  \vspace{2pt}
  \caption{Even more numbers in a big table are shown here. This table spans across the full page, text width plus margin.\label{table3}}%
\end{table*}

%
\section{More information}
\marginpar{It is also possible to put a comment in the margin without a corresponding mark in the text with \texttt{marginpar}.} This is a short example file to show the features of the caesar class together with the sidenotes package. Sometimes it is necessary to compile the document up to 3 times in order to get the alignment of all objects correctly.
%
% \printbibliography[heading=bibintoc]
 %
\end{document}
